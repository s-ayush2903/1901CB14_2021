\documentclass[12pt,  letterpaper,  twoside]{article}
\usepackage[utf8]{inputenc}
\usepackage{xcolor}
\usepackage{hyperref}
\usepackage[margin=1.5cm]{geometry}
\hypersetup{colorlinks=true,linkcolor=blue,urlcolor=blue}

\title{Mid Sem Assignment CS384 -  CPI SPI Generator}
\author{Dr. Mayank Agarwal}
\date{Assignment Given: 29th Sep 2021,\\ Deadline 3rd Oct 2021,  
23:59\\Submission: GitHub }

\begin{document}
	\maketitle  
	\textbf{Things to be kept in mind}\\
	\begin{enumerate}
%		\item Dont take any inputs from user . 
				\item You \textcolor{red}{cannot }use \textbf{pandas} 
library.   
\item Program will be checked for plagiarism.   

	\end{enumerate}
	
	
You are given a \textbf{grades.csv} file that contains data for the grades for 
\textbf{IIT 
Antartica}. 	The content of the file are: \\
\textbf{Roll}: Roll of the student\\	
\textbf{Sem}: Semester in which that sub is studied	\\
\textbf{SubCode}: Course Code \\	
\textbf{Credit}: Sub Credits\\	
\textbf{Grade}: Obtained Grade\\	
\textbf{Sub\_Type}: Whether Core or Elective etc \\

\textbf{names-roll.csv} - This contains the mapping of names and roll numbers.\\

\textbf{subjects\_master.csv} - This contains mapping of course codes and the 
name of the subject. 

Your task is to generate a marksheet of every roll number and save as ".xlsx" 
file in the output folder. A sample ``0401CS02.xlsx" is provided for your 
reference. Its self explainable. The names of each sheet should be like the 
ones mentioned ``0401CS02.xlsx". Names: Overall, Sem1,...,SemN 



Calculation of SPI and CPI:
Suppose in a given semester a student has taken four courses having credits C1, 
C2, C3 and C4 and
grade points in those courses are G1, G2, G3 and G4 respectively. Then,
\begin{equation}
	SPI = (C_1 * G_1 + C_2 * G_2 + C_3 * G_3 + C_4 * G_4) / (C_1 + C_2 + C_3 + 
	C_4)
\end{equation}

\begin{equation}
	CPI = (SPI1 * Credits\ in\ semester_1 + SPI2 * Credits\ in\ semester_2 + 
	...) / (Total\ credits)
\end{equation}


For example, if in a given semester a student has taken four courses having 
credits 6, 6, 6, and 8
and grade points in those courses are 10, 9, 8, 6 respectively. Then,
\begin{equation}
	SPI = (6 * 10 + 6 * 9 + 6 * 8 + 6 * 6) / (6 + 6 + 6 + 8) = 7.62
\end{equation}




If the student has an SPI of 7.62 in the 1st semester worth (say) 32 credits 
and 8.2 in the next
semester worth 36 credits,


\begin{equation}
	CPI (at\ the\ end\ of\ 2nd\ semester) = (7.62 * 32 + 8.2 * 36) / (32 + 36) 
	= 7.93
\end{equation}





Grade Numeric Equivalent:
\begin{itemize}
	\item AA - 10
	\item AB - 9
	\item BB - 8
	\item BC - 7
	\item CC - 6
	\item CD - 5
	\item DD - 4
	\item F - 0
	\item I - 0
\end{itemize}
	
\end{document}